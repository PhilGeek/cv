%!TEX TS-program = xelatex 
%!TEX encoding = UTF-8 Unicode
%!TEX TS-program = xelatex 
%!TEX TS-options = -output-driver="xdvipdfmx -q -E"
%!TEX encoding = UTF-8 Unicode
%
%  cv
%
%  Created by Mark Eli Kalderon on 2008-07-21.
%

\documentclass[contbibnum]{cv}

\usepackage{geometry} \geometry{a4paper} 
\usepackage{pdfsync} 
\usepackage{url}
\usepackage[round]{natbib}
\usepackage{fontspec,xltxtra,xunicode}
\defaultfontfeatures{Scale=MatchLowercase,Mapping=tex-text}
\setmainfont{Hoefler Text}
\setsansfont{Gill Sans}
\setmonofont{Inconsolata}

%% You can modify the fonts used in the document be using the
%% following macros. They take one parameter which is the font
%% changing command.
%% \headerfont: the font used in both headers.
%%              Defaults to sans serif.
%% \titlefont:  the font used for the title.
%%              Defaults to \LARGE sans-serif semi bold condensed.
%% \sectionfont: the font used by \section when beginning a new topic.
%%              Defaults to sans-serif semi bold condensed.
%% \itemfont:   the font used in descriptions of items.
%%              Defaults to sans-serif slanted.
% to make your name even bigger, uncomment the following line:
% \titlefont{\Huge}
%%
%% You can modify the following parameters using \setlength:
%% \topicmargin: the left margin inside topics.
%%               Defaults to 20% of the text width (0.20\textwidth).
% To get more room for left column of Topic layouts, uncomment following line:
% \renewcommand{\topicmargin}{0.25\textwidth}

\renewcommand{\topicmargin}{0.30\textwidth}

\begin{document}

% \leftheader{Department of Philosophy\\
% University College London\\
% Gower Street\\
% London WC1E 6BT}
% 
% \rightheader{TEL (0)20 7679-3577\\
% EMAIL \href{mailto:m.kalderon@ucl.ac.uk}{m.kalderon@ucl.ac.uk}\\
% WEB \url{http://markelikalderon.com}}

\title{Mark Eli Kalderon}

\maketitle

% I am an analytic philosopher teaching at University College London. I am also the current editor of the \emph{Proceedings of the Aristotelian Society}. I received my PhD from Princeton in 1995. I have taught at University of California Riverside, Princeton, UCLA, and Caltech. According to the present \emph{deformation professionnelle} my work is in M\&E (metaphysics and epistemology). My current research concerns color, consciousness, and metaethics.
In support of case for promotion to Professor

\section{Personal Details}\label{sec:personal_details} % (fold)

\begin{topic}
	\item[Name] Mark Eli Kalderon
	\item[Department] Philosophy
	\item[Present Appointment] Reader
	\item[Date of Appointment] 2004
\end{topic}

% section personal_details (end)

\section{Education}\label{sec:education} % (fold)

\begin{topic}
	\item[1995] Princeton University, PhD
	\item[1987] University of Michigan, Ann Arbor, no degree
\end{topic}

% section education (end)

\section{Professional History}\label{sec:professional_history} % (fold)

\begin{topic}
	\item[2004--present] Reader, University College London
	\item[2000--2004] Lecturer, University College London
	\item[1998--2000] Weisman Postdoctoral Instructor in Philosophy, California Institute of Technology
	\item[1997--1999] Lecturer, University of California, Los Angeles
	\item[1996--1997] Lecturer, Princeton University
	\item[1995--1996] Visiting Assistant Professor, University of California, Riverside
\end{topic}

% section professional_history (end)

\section{Other Appointments and Affiliations}\label{sec:other_appointments_and_affiliations} % (fold)

\begin{topic}
	\item[2005--present] Editor, \emph{Proceedings of the Aristotelian Society}, \emph{Proceedings of the Aristotelian Society, Supplementary Volume}
	\item[2000--2005] Associate Editor, \emph{Mind}
	\item[2000--present] Member, Mind Association
	\item[2000--present] Member, Aristotelian Society
	\item[2003--present] Member, British Philosophical Association
	\item[1995--present] Member, American Philosophical Association
	\item[1995--present] I have refereed for the following journals: \emph{Australasian Journal of Philosophy}, \emph{Canadian Journal of Philosophy} \emph{Dialectica}, \emph{Mind}, \emph{Noûs}, \emph{Philosophical Studies}, \emph{Philosophy and Phenomenological Research}, \emph{Ratio}
\end{topic}

% section other_appointments_and_affiliations (end)

\section{Grants}\label{sec:grants} % (fold)

\begin{topic}
	\item[2003] AHRC Research Leave Scheme, Project Title: \emph{Moral Fictionalism}
\end{topic}

% section grants (end)

\section{Invited Talks}\label{sec:presentations} % (fold)

\begin{topic}
    \item[March 1995] University of Pittsburgh, ``What Numbers Could Be''
    \item[November 1995] University of California, Riverside, ``What Numbers Could Be''
    \item[April 1996] University of California, Riverside, ``Inferential Role Semantics and Logic''
    \item[September 1996] Czech Academy of Sciences, ``The Transparency of Truth''
    \item[December 1997] University of California, Riverside, ``Revelation, Necessity, and Objective Color''
    \item[February 1998] Arizona State University, ``Color and the Inverted Spectrum''
    \item[February 1998] University of Colorado, Boulder, ``Color and the Inverted Spectrum''
    \item[September 2000] University of California, Los Angeles, ``Color and the Inverted Spectrum''
    \item[April 2000] Pacific APA, ``Comments on Cunningham and Emergence''
    \item[November 2000] University of Manchester, ``On the Ignorance of Archangels''
    \item[June 2001] University of London, School of Advanced Study, Philosophy Programme, ``On the Ignorance of Archangels''
    \item[August 2001] Interuniversity Center, Dubrovnik, ``Old Facts Newly Known''
    \item[November 2001] University of St Andrews, ``Modal Structuralism and Cardinality''
    \item[April 2002] Pacific APA, ``The Open Question Argument, Frege's Puzzle, and Leibniz's Law''
    \item[August 2002] Interuniversity Center, Dubrovnik, ``Selection and Manifest Color'' 
    \item[March 2003] University of London, School of Advanced Study, Philosophy Programme, ``Moral Fictioanlism and the Pragmatic Fallacy''
    \item[November 2003] Unversity College London, ``Moral Pyrronhism and Noncognitivism''
    \item[June 2003] Birkbeck, ``Moral Pyrronhism and Noncognitivism''
    \item[July 2003] Joint Sessions, ``Selection and Manifest Color''
    \item[February 2004] Oxford University, ``Moral Pyrronhism and Noncognitivism''
    \item[May 2004] University of York, ``Moral Pyrronhism and Noncognitivism''
    \item[June 2004] University of London, School of Advanced Study, Philosophy Programme, ``How Not to be a Normative Irrealist''
    \item[November 2004] Cambridge Moral Sciences Club, ``Cognitive Virtue and Publicity''
    \item[January 2005] SOFIA XVI, ``Groundwork for a Nonconcessive Expressivism''
    \item[February 2005] University of Bristol, ``Cognitive Virtue and Publicity''
    \item[February 2005] University of London, School of Advanced Study, Philosophy Programme, ``Comments on Caplan and Heck''
    \item[May 2005] University of Leeds, ``Moral Pyrronhism and Nonconcessive''
    \item[October 2005] University of Sheffield, ``Color Pluralism and the Location Problem''
    \item[November 2005] University of Nottingham, ``Color Pluralism and the Location Problem''
    \item[May 2006] University of London, School of Advanced Study, Institute of Philosophy, ``Respecting Value''
    \item[May 2006] Universty of Fribourg, ``Cognitive Virtue and Publicity''
    \item[November 2006] University of London, School of Advanced Study, Institute of Philosophy, ``Epistemic Relativism''
    \item[May 2007] University of St Andrews, ``Moral Fictionalism, the Frege-Geach Problem, and Reasonable Inference''
    \item[May 2007] University of Oxford, ``Metamerism, Constancy, and Knowing Which''
    \item[May 2007] European Forum for Philosophy, ``Moral Fictionalism''
    \item[June 2007] University of Leeds, ``Kendall Walton's Influence on Metaphysics''
    \item [October 2007] University College London, ``Metamerism, Constancy, and Knowing Which''
    \item[October 2007] University of Nottingham, ``The Multiply Qualitative''
    \item[November 2007] Birkbeck Colege, ``The Multiply Qualitative''
    \item[January 2008] Presentation to Anthony Eagle's fictionalism seminar, Oxford University, ``Moral Fictionalism, the Frege-Geach Problem, and Reasonable Inference''
    \item[February 2008] Reading University, ``Metamerism, Constancy, and Knowing Which''
    \item[October 2008] Warwick University, ``Color and the Problem of Perceptual Presence''
\end{topic}

\section{Academic Supervision}\label{sec:academic_supervision} % (fold)



% section academic_supervision (end)

\section{Research Activity}\label{sec:research_activity} % (fold)

\subsection{Philosophy of Mathematics}\label{sub:philosophy_of_mathematics} % (fold)



% subsection philosophy_of_mathematics (end)

\subsection{Ethics}\label{sub:ethics} % (fold)



% subsection ethics (end)

\subsection{Color}\label{sub:color} % (fold)

The nature of color and color experience has been a persistent interest since graduate school where I attended seminars on these topics by Mark Johnston and Saul Kripke. I have been keen to defend the view that colors are mind-independent qualities of material objects as they seem, pre-theoretically to be, and that the phenomenology of color experience can be explained in terms of the presentation of these mind-independent qualities in color experience. 

``Color and the Inverted Spectrum'', 2000, co-written David Hilbert, addresses the so-called inverted spectrum argument. The inverted spectrum argument uses actual and hypothetical variation in color perception to establish a claim about the nature of color experience---that there could be a phenomenological difference between color experiences without a difference in what's presented. We criticize then extant versions of the argument and further argue that any version of the argument will fail to account adequately for asymmetries in the phenomenological color space.

Perceptual variation has been deployed not only to establish claims about color experience but also to establish claims about the nature of the colors. The argument from conflict appearances uses perceptual variation to argue against the reality of the colors. Most ways of resisting this argument do so at the cost of denying that the colors are mind-independent. In ``Color Pluralism'', 2007, I argue that the mind-independence of the colors can be retained consistent with veridical perceptual variation if we abandon color monism for a version of color pluralism. According to color pluralism, there are a plurality of families of color properties perceptually available to different kinds of perceivers. (So, for example, bees see a different family of colors than humans.)

``Metamerism, Constancy, and Knowing Which'', forthcoming, is a direct argument that color experience has a presentational phenomenology. I argue that color experience must have a presentational phenomenology given their epistemic role in identifying colors across different circumstances of perception. I also argue that dispositional theories cannot adequately account for color constancy.

``The Multiply Qualitative'' takes up Sydney Shoemaker's criticism of ``Color and the Inverted Spectrum'' in his paper ``Color, Content, and Character''. I argue that the causal intuitions that Shoemaker appeals to fail if experience is conceived to be relational---the natural metaphysics of experience understood as having a presentational phenomenology.

``Color and the Problem of Perceptual Presence'', submitted, is an extension of the material in ``Metamerism, Constancy, and Knowing Which''. Like dispositional theories of color, Noë's phenomenal objectivism attempts to understand being colored in terms of looking colored. The account is meant to resolve a prima facie conflict in or between experiences in cases of color constancy. I argue that no prima facie conflict is generated and that phenomenal objectivism, like the dispositional theory, fails to provide an adequate account of color constancy.

``Color Illusion'', currently being composed. As standardly conceived, an illusion is an experience of an object \( o \) appearing \( F \) where \( o \) is not in fact \( F \). Paradigm examples of color illusion, however, do not fit this pattern. A diagnosis of this uncovers different sense of appearance talk that is the basis of a dilemma for the standard conception. The dilemma is only a challenge. But if the challenge cannot be met, then any conception of experience, such as representationalism, that is committed to the standard conception is false. I argue that a relational theory of color experience provides a better account of color illusion.

I have applied for a Leverhulme Major Research Fellowship to pursue this research further. I proposed to write a book, ``Regaining Color: From Inverted Spectra to Conflicting Appearances''. Two problems form the intellectual background of the present project---the problem of the manifest and the hard problem of consciousness. 

The first is the special and central case of the problem of the manifest: How can the colors, given their perceived qualitative nature, be materially realized?

The second is a contemporary problem that lacks ancient precedent---the so-called hard problem of consciousness. Whereas the problem of the manifest is a problem in metaphysics, the hard problem of consciousness is a problem in the philosophy of mind: How can experience, given its qualitative character, be materially realized? And just as the fundamental problem in the metaphysics of color is a special case of the problem of the manifest, a problem about color experience is a special case of the hard problem of consciousness: How can color experience, given its qualitative character, be materially realized?

So formulated, these problems are structurally parallel. This may be no accident. Indeed, I believe that there is an explanation for this. Consider one way of responding to the problem of the manifest as it arises for the colors. The metaphysical problem only arises if the qualitative nature of the colors is part of what is present in color experience. But, suppose, instead, that the qualitative character is not a matter of \emph{what} is present in color experience but \emph{how} the colors are presented in experience. So understood, the problematic qualitative character is, strictly speaking, not a feature of the colors (and, hence, no obstacle to their material instantiation) but a feature of color experience. However, such a reply immediately faces the hard problem of consciousness, for if we conceive of the mind as part of nature, we now have the problem of how color experience, given its qualitative character, might be materially realized. It can seem that we have only pushed the problem back a level.

Supppose, however, we could successfully defend the idea that the qualitative character of color experience is best explained by what's present in color experience. Then the hard problem of consciousness as it arises with respect to color experience has \emph{dissolved} into the metaphysics of color. Indeed, the hard problem of consciousness rests on a kind of \emph{introjective error}---it arises from systematically misattributing mind-independent qualities of material things to mental things. In wondering how the experience of red, given its qualitative character, could be materially realized by the visual system of the perceiver, one attributes to something mental, the conscious experience of red, the qualitative nature of a mind-independent color.

% subsection color (end)

% section research_activity (end)

\section{Teaching Activity}\label{sec:teaching_activity} % (fold)



% section teaching_activity (end)

\section{Knowledge Transfer}\label{sec:knowledge_transfer} % (fold)

\begin{topic}
	\item[2006-present] Excursus, a blog on the technology of writing. Advocating the use of plain text over propriety binaries (such as Word documents) it presents a series of tutorials on how to adapt various tools for programmers to the task of academic writing.
	\item[2008] Wrote introductory remarks to \emph{Pleasure Island}, a collection of photographs by Jocelyn Bain Hogg of the Ibiza Rocks concert.
	\item[2008] Interview with IC Radio on seeing red 
	\item[2007] Wrote and cowrote three articles for \emph{PracTeX} on using version control in the production of LaTeX documents
	\item[2006] Consultant for Kan Akita Design Office on philosophy and sustainable design
	\item[2004] Consultant for Kan Akita Design Office on philosophy and sustainable design
	\item[2001] Interview with the SciFi channel about Nietzsche and the nature of violence
\end{topic}

% section knowledge_transfer (end)

\section{Enabling Activity}\label{sec:enabling_activity} % (fold)

\begin{topic}
    \item[2007--present] Graduate Tutor
    \item[2004--2006] Graduate Tutor
    \item[2004--present] Graduate Committee
    \item[2004--2008] Subject Panel in Philosophy
    \item[2002--2004] Undergraduate Admissions Tutor
    \item[2001--present] Placement Officer
\end{topic}

% section enabling_activity (end)

\section{Five Most Significant Publications}\label{sec:five_most_significant_publications} % (fold)

% Color and the Inverted Spectrum
% The Transparency of Truth
% Moral Fictionalism
% Color Pluralism

% section five_most_significant_publications (end)



\nocite{Kalderon:1987lr}
\nocite{Kalderon:1996fk}
\nocite{Kalderon:1997lr}
\nocite{Kalderon:2000qy}
\nocite{Kalderon:2001qf}
\nocite{Kalderon:2004lr}
\nocite{Kalderon:2005gz}
\nocite{Kalderon:2005lr}
\nocite{Kalderon:2006fk}
\nocite{Kalderon:2006tg}
\nocite{Kalderon:2007fk}
\nocite{Kalderon:2008qy}
\nocite{Kalderon:2006fk}
\nocite{Kalderon:2007lr}
\nocite{Kalderon:2007mr}
\nocite{Kalderon:2007gk}
\nocite{Kalderon:2007qy}
\nocite{Skiadas:2007lr}
\nocite{Hilbert:2000on}
\nocite{Kalderon:2008ud}
\nocite{Kalderon:2008mi}

\bibliographystyle{plainnat} 
\bibliography{Philosophy}


\end{document}
