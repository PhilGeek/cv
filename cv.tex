%%% Mark Eli Kalderon 

\documentclass[11pt]{article}

%%%------------------------------------------------------------------------
%%% Metadata
%%%------------------------------------------------------------------------

%% Change as needed. Or just add me as a coauthor. Only some of these are 
%% used below in the hyperref declaration and address banner section.
\def\myauthor{Mark Eli Kalderon}
\def\mytitle{Vita}
\def\mycopyright{\myauthor}
\def\mykeywords{}
\def\mybibliostyle{plain}
\def\mybibliocommand{}
\def\mysubtitle{}
\def\myaffiliation{University College London}
\def\myaddress{Department of Philosophy}
\def\myemail{m.kalderon@ucl.ac.uk}
\def\myweb{http://markelikalderon.com}
\def\myphone{020 7697 3577}
\def\myfax{020 7209 0554}
\def\myversion{}
\def\myrevision{}


\date{} % not used (revision control instead)
\def\mykeywords{Mark Eli Kalderon, Kalderon CV, Resume, Philosophy}

%%%------------------------------------------------------------------------  
%%% Git version tracking 
%%%------------------------------------------------------------------------

%% If you don't use git or the vc package (from CTAN), comment this out.
%% If you comment it out, be sure to remove the \rfoot comment below, too.
% \input{vc}

%%%------------------------------------------------------------------------
%%% Required style files
%%%------------------------------------------------------------------------
\usepackage{url,fancyhdr}
%%\usepackage{revnum} % for reverse-numbered publications (revnumerate environment) if needed.

%% needed for xelatex to work
\usepackage{fontspec}
\usepackage{xunicode}

%% color for the links 
% \usepackage[usenames,dvipsnames]{color}

%% hyperlinks
\usepackage[xetex, 
	% colorlinks=true,
	% urlcolor=Gray,
	plainpages=false,
  	pdfpagelabels,
  	bookmarksnumbered,
  	pdftitle={\mytitle},
  	pagebackref,
  	pdfauthor={\myauthor},
  	pdfkeywords={\mykeywords}
  	]{hyperref}

%%%------------------------------------------------------------------------
%%% Document
%%%------------------------------------------------------------------------
\begin{document}

%% Choose fonts for use with xelatex
%% Minion and Myriad are widely available, from Adobe. 
%% Pragmata is available to buy at http://www.fsd.it/fonts/pragma.htm
%% and is worth every penny. Any good monospace font will work fine, though.
%% Consolas or inconsolata are good alternatives.
\setromanfont[Mapping={tex-text},Numbers={OldStyle},Ligatures={Common}]{Hoefler Text} 
\setsansfont[Mapping=tex-text,Colour=AA0000]{Gill Sans}
\setmonofont[Mapping=tex-text,Scale=0.9]{Inconsolata} 


%%%------------------------------------------------------------------------
%%% Local commands
%%%------------------------------------------------------------------------

%% Marginal header
%% Note: as the document goes on you may need to introduce a (gradually increasing)
%% \vspace element to keep the marginal header pleasingly aligned with the first 
%% item in the body text. Like this: \marginhead{{\vskip 0.4em}Grants}, or 
%% \marginhead{{\vskip 0.8em}Service}. Experiment as needed.
\newcommand{\marginhead}[1]{\marginpar{\textsf{{\footnotesize\vspace{-1em}\flushright #1}}}}


%% custom ampersand (font consistent with the one chosen above)
% \newcommand{\amper}{{\fontspec[Scale=.95,Colour=AA0000]{Minion Pro Medium}\selectfont\&\,}}

%% No bullets on labels
\renewcommand{\labelitemi}{~} 

%% Custom hanging indent for vita items
\def\ind{\hangindent=1 true cm\hangafter=1 \noindent}
%\def\ind{\hangindent=18pt\hangafter=1 \noindent}
\def\labelitemi{~}
\renewcommand{\labelitemii}{~}

%%%------------------------------------------------------------------------
%%% Page layout
%%%------------------------------------------------------------------------
\pagestyle{fancy}
\renewcommand{\headrulewidth}{0pt}
\fancyhead{}
\fancyfoot{}
\rhead{{\scriptsize\thepage}}

%% git revision control footer 
% \rfoot{\texttt{\scriptsize \VCRevision\ on \VCDateTEX}} % git revision info inserted via external script -- see docs for vc package for details. comment out this line if you're not using vc, and also remove the \input{vc} line above.

%%%------------------------------------------------------------------------
%%% Address and contact block
%%%------------------------------------------------------------------------
\begin{minipage}[t]{2.95in}
 \flushright {\footnotesize \href{http://ucl.ac.uk/philosophy/}{Department of Philosophy} \\ University College London, \\ \vspace{-0.05in} London, \textsc{wc1e 6bt}}
  
\end{minipage}
\hfill     
%\begin{minipage}[t]{0.0in}
% dummy (needed here)
%\end{minipage}
\hfill
\begin{minipage}[t]{1.7in}
  \flushright \footnotesize Phone: \myphone \\ 
  Fax: \myfax  \\ 
  {\scriptsize  \texttt{\href{mailto:\myemail}{\myemail}}} \\
  {\scriptsize  \texttt{\href{\myweb}{\myweb}}}
\end{minipage}


\medskip

%% Name 
\noindent{\Large {\textsc{Mark Eli Kalderon}}}
\reversemarginpar

\medskip       

%% Appointments
\medskip
\marginhead{Appointments}

\noindent\emph{University College London \vspace{0.01in}}

\ind 2009--Present. Professor, Department of Philosophy.      

\ind 2004--2009. Reader, Department of Philosophy.

\ind 2000--2004. Lecturer, Department of Philosophy.

\medskip
\noindent\emph{California Institute of Technology \vspace{0.01in}}

\ind 1998--2000. Weisman Postdoctoral Instructor in Philosophy.

\medskip
\noindent\emph{UCLA \vspace{0.01in}}

\ind 1997--1999. Lecturer in Philosophy.

\medskip
\noindent\emph{Princeton University \vspace{0.01in}}

\ind 1996--1997. Lecturer in Philosophy.

\medskip
\noindent\emph{University of California, Riverside \vspace{0.01in}}

\ind 1996--1997. Visiting Assistant Professor in Philosophy.


\bigskip

%% Education

\marginhead{Education}

\noindent\emph{Princeton University \vspace{0.01in}}

\ind 1995. PhD, Philosophy.

\medskip
\noindent\emph{University of Michigan, Ann Arbour\vspace{0.02in}}

\ind 1987. BA, Philosophy

\bigskip
 
%% Publications
%%%% Book
\marginhead{{\vskip 0.3em}Publications}
\medskip
\noindent\emph{Books \vspace{0.01in}}

\ind  Mark Eli Kalderon. 2005. \emph{\href{http://ukcatalogue.oup.com/product/9780199275977.do}{Moral Fictionalism}}. Oxford:~Oxford University Press. \vspace{-0.075in}

%%%% Edited Volumes
\medskip
\noindent\emph{Edited Volumes \vspace{0.01in}}

\ind Mark Eli Kalderon. 2005. \emph{\href{http://ukcatalogue.oup.com/product/9780199282180.do}{Fictionalism in Metaphysics}} Oxford:~Oxford University Press. \vspace{-0.075in}
 
\normalsize

\bigskip
\noindent\emph{Journal articles \vspace{0.05in}}
 
%% Use revnumerate environment if numbered publications are needed. 
%% (Include it above in the preamble).
%% \renewcommand{\labelenumi}{\textsc{a}\theenumi.}
%% \begin{revnumerate}

\ind Mark Eli Kalderon. 2013. ``\href{http://dx.doi.org/10.1093/analys/ans147}{Does Metaethics Rest on a Mistake?}'' \emph{Analysis} 73(1) 129--138

\ind Mark Eli Kalderon. 2011. ``\href{http://mind.oxfordjournals.org/content/120/478/239.full.pdf+html}{The Multiply Qualitative}'' \emph{Mind} 120(478):~239--262. 

\ind Mark Eli Kalderon. 2011. ``\href{http://onlinelibrary.wiley.com/doi/10.1111/j.1468-0068.2010.00781.x/pdf}{Color Illusion}.''
 \emph{No{\^u}s} 45(4):~751--775.

\ind Mark Eli Kalderon. 2009. ``\href{http://philreview.dukejournals.org/content/118/2/225.full.pdf+html}{Epistemic Relativism}.'' \emph{The Philosophical Review} 10:~478--500. 

\ind Mark Eli Kalderon. 2008. ``\href{http://onlinelibrary.wiley.com/doi/10.1111/j.1467-8284.2007.00728.x/full}{Moral Fictionalism, the Frege-Geach Problem, and Reasonable Inference}'' \emph{Analysis} 68(2):~133--143. 


\ind Mark Eli Kalderon. 2008. ``\href{http://onlinelibrary.wiley.com/doi/10.1111/j.1468-0149.2008.454_1.x/pdf}{Moral Fictionalism, Summary}.'' \emph{Philosophical Books} 49(1):~1--3.


\ind Mark Eli Kalderon. 2008.  ``\href{http://onlinelibrary.wiley.com/doi/10.1111/j.1468-0149.2008.454_5.x/abstract}{The Trouble with Terminology}.'' \emph{Philosophical Books} 49(1):~33--41.


\ind Mark Eli Kalderon. 2008. ``\href{http://onlinelibrary.wiley.com/doi/10.1111/j.1468-0378.2008.00324.x/pdf}{Respecting Value}''  \emph{European Journal of Philosophy}
16(3):~341--365.


\ind  Mark Eli Kalderon. 2008.  ``\href{http://mind.oxfordjournals.org/content/117/468/935.full.pdf+html}{Metamerism, Constancy, and Knowing Which}.'' \emph{Mind} 117(468):~935--971. 

\ind  Mark Eli Kalderon. 2007.  ``\href{http://philreview.dukejournals.org/content/116/4/563.full.pdf+html}{Color Pluralism}.'' \emph{The Philosophical Review} 116(4):~563--601.

\ind  Mark Eli Kalderon. 2004.  ``\href{http://onlinelibrary.wiley.com/doi/10.1111/j.1933-1592.2004.tb00341.x/pdf}{Open Questions and the Manifest Image}.'' \emph{Philosophy and Phenomenological Research} 68(2):~251--289.

\ind  Mark Eli Kalderon. 2001.  ``\href{http://www.springerlink.com/content/u5286534982v0317/fulltext.pdf}{Reasoning and Representing}.'' \emph{Philosophical Studies} 105:~129--160.

\ind Mark Eli Kalderon. 1997.
``\href{http://mind.oxfordjournals.org/content/106/423/475.full.pdf}{The Transparency of Truth}.'' \emph{Mind} 106(423): 475--497.

\ind Mark Eli Kalderon. 1996.
``\href{http://philmat.oxfordjournals.org/content/4/3/238.full.pdf}{What Numbers Could Be (And, Hence, Necessarily Are)}.'' \emph{Philosophia Mathematica} 4:238--255.

\ind Mark Eli Kalderon. 1987.
``\href{http://www.jstor.org/stable/4319905}{Epiphenomenalism and Content}.'' \emph{Philosophical Studies} 52(1):71--90.

%\end{revnumerate}
%\newpage
\bigskip

\noindent\emph{Book chapters \vspace{0.05in}}
% \renewcommand{\labelenumi}{\textsc{c}\theenumi.}
% \begin{revnumerate}

\ind Mark Eli Kalderon. Forthcoming. ``Experiential Pluralism and the Power of Perception,'' in \emph{Themes from Charles Travis}. A discussion of the metaphysics of perceptual capacities. I argue that a pattern of dependence obtains between a perceptual capacity and its exercise that is inconsistent with experiential monism, the doctrine that all sense experience has a single common nature.

\ind Mark Eli Kalderon and Charles Travis. Forthcoming. ``Oxford Realism,''  in \emph{Oxford Handbook of the History of Analytic Philosophy}, edited by Michael Beany. Oxford: Oxford University Press.

\ind Mark Eli Kalderon. 2011. ``Before the Law'', in \emph{Philosophical Issues, The Epistemology of Perception}, edited by Berit Brogaard. Oxford: Blackwell Publishing.

\ind David Hilbert and Mark Eli Kalderon. 2000. ``\href{http://markelikalderon.com/wp-content/uploads/2006/09/cis.pdf}{Color and the Inverted Spectrum},'' in \emph{Color Perception: Philosophical, Psychological, Artistic, and Computational Perspectives}, volume~9 of \emph{Vancouver Studies in Cognitive Science}, edited by Steven Davies. Oxford: Oxford University Press.

%\end{revnumerate}

\bigskip 

 
%\newpage
\noindent\emph{In preparation \vspace{0.05in}}

%\renewcommand{\labelenumi}{\textsc{r}\theenumi.}
%\begin{revnumerate}

\ind Mark Eli Kalderon. Forthcoming, \emph{Dialectica}. ``Color and the Problem of Perceptual Presence''. Like dispositional theories of color, Noë's phenomenal objectivism attempts to understand being colored in terms of looking colored. The account is meant to resolve a prima facie conflict in or between experiences in cases of color constancy. I argue that no prima facie conflict is generated and that phenomenal objectivism, like the dispositional theory, fails to provide an adequate account of color constancy.

\ind Mark Eli Kalderon. Under review. \emph{Form Without Matter: Empedocles and Aristotle on Color Perception}. The monograph is an essay in the philosophy of perception written in the medium of historiography. I argue that a puzzle in Empedocles (and one that persists to this day) motivates Aristotle's doctrine that sense perception receives the form without the matter of the perceived object. Currently under review with Oxford University Press.

\ind Mark Eli Kalderon. Unpublished manuscript. ``Sights and Sounds.'' A methodological discussion of visuocentrism. I argue that O'Callaghan's criticism of the wave view of sound is only plausible if audition is more like vision than it actually is.

\ind Mark Eli Kalderon. Unpublished Manuscript. ``Realism and Perceptual Appearance.'' In his 1904 letter to G.F. Stout, Cook Wilson distinguishes objective and subjective conceptions of appearance, and provides a diagnosis for the modern acceptance of the subjective conception in terms of a confused misdescription of the objective appearances that perceptual experience affords. Moreover, Cook Wilson links subjective appearances with idealism, the suggestion being that perceptual appearances must be objective if they are to afford us with something akin to proof of a world without the mind.

% %\end{revnumerate}
 \bigskip

%% Presentations
\marginhead{{\vskip 0.4em}Invited Talks}
\medskip

\ind 2013. ``Experiential Pluralism and the Power of Perception''. Cumberland Lodge, Warwick Philosophy Department, February.

\ind 2012. ``Experiential Pluralism and the Power of Perception''. \emph{Themes from Charles Travis}, University of East Anglia, June

\ind 2012. ``Experiential Pluralism and the Power of Perception''. \emph{Logos}, University of Barcelona, May.

\ind 2011. ``The Generation of the Hues''. University of Warwick, December.

\ind 2011. ``The Generation of the Hues''. University of Aberdeen, November.

\ind 2011. ``The Generation of the Hues''. Jowett Society, University of Warwick, Oxford, October.

\ind 2011. ``Before the Law''. University of Cork, October.

\ind 2011. ``Aristotle on Transparency''. UCL, October.

\ind 2011. ``Aristotle on Transparency'', Interuniversity Centre, Dubrovnik, July.

\ind 2011. ``Before the Law'', Inaugural Lecture, UCL, May.

\ind 2008.  ``Color and the Problem of Perceptual Presence''. Warwick University, May.

\ind 2008. ``Metamerism, Constancy, and Knowing Which''. Reading University, January.

\ind 2008.``Moral Fictionalism, the Frege-Geach Problem, and Reasonable Inference''. Presentation to Anthony Eagle's fictionalism seminar, Oxford University, January.

\ind 2007. ``The Multiply Qualitative''. Birkbeck Colege, November.

\ind 2007. ``The Multiply Qualitative''. University of Nottingham, October.

\ind 2007. ``Metamerism, Constancy, and Knowing Which''. University College London, October.

\ind  2007. ``Kendall Walton's Influence on Metaphysics''. University of Leeds, May.

\ind 2007. ``Moral Fictionalism''. European Forum for Philosophy, May.

\ind 2007. ``Metamerism, Constancy, and Knowing Which''. Jowett Society, University of Oxford, May.

\ind 2007. ``Moral Fictionalism, the Frege-Geach Problem, and Reasonable Inference''. University of St Andrews, May.

\ind 2006. ``Epistemic Relativism''. University of London, School of Advanced Study, Institute of Philosophy, November.

\ind 2006. ``Cognitive Virtue and Publicity'' Universty of Fribourg, May. 

\ind 2006. ``Respecting Value''. University of London, School of Advanced Study, Institute of Philosophy, May.

\ind 2005. ``Color Pluralism and the Location Problem''. University of Nottingham, November.

\ind 2005. ``Color Pluralism and the Location Problem'' University of Sheffield, October.

\ind 2005. ``Moral Pyrronhism and Nonconcessive''. University of Leeds, May.

\ind 2005. ``Comments on Caplan and Heck''. University of London, School of Advanced Study, Philosophy Programme, February.

\ind 2005. ``Cognitive Virtue and Publicity''. University of Bristol, January.

\ind 2005.  ``Groundwork for a Nonconcessive Expressivism''. SOFIA XVI, January.

\ind 2004. ``Cognitive Virtue and Publicity''. Cambridge Moral Sciences Club, November.

\ind  2004. ``How Not to be a Normative Irrealist''. University of London, School of Advanced Study, Philosophy Programme, June.

\ind 2004. ``Moral Pyrronhism and Noncognitivism''. University of York, May.

\ind 2004. ``Moral Pyrronhism and Noncognitivism'' Oxford University, February.

\ind  2003. ``Selection and Manifest Color''. Joint Sessions, July.

\ind  2003. ``Moral Pyrronhism and Noncognitivism''. Birkbeck, June.

\ind 2003. ``Moral Pyrronhism and Noncognitivism''. Unversity College London, November.

\ind 2003. ``Moral Fictionalism and the Pragmatic Fallacy''. University of London, School of Advanced Study, Philosophy Programme, March.

\ind 2002. ``Selection and Manifest Color''. Interuniversity Center, Dubrovnik, August.

\ind 2002. ``The Open Question Argument, Frege's Puzzle, and Leibniz's Law''. Pacific APA, April.

\ind 2001. ``Modal Structuralism and Cardinality''.  University of St Andrews, December.

\ind 2001. ``Old Facts Newly Known''. Interuniversity Centre, Dubrovnik, August.

\ind 2001. ``On the Ignorance of Archangels''. University of London, School of Advanced Study, Philosophy Programme, June.

\ind 2000. ``On the Ignorance of Archangels''. University of Manchester, November.

\ind 2000. ``Comments on Cunningham and Emergence''. Pacific APA, April.

\ind 2000. ``Color and the Inverted Spectrum'' University of California, Los Angeles, December.

\ind 1998. ``Color and the Inverted Spectrum''. University of Colorado, Boulder, February.

\ind 1998. ``Color and the Inverted Spectrum''. Arizona State University, February.

\ind 1997. ``Revelation, Necessity, and Objective Color''. University of California, Riverside, December.

\ind 1996. ``The Transparency of Truth''. Czech Academy of Sciences, September.

\ind 1996. ``Inferential Role Semantics and Logic''. University of California, Riverside, April.

\ind 1995. ``What Numbers Could Be''. University of California, Riverside, November.

\ind 1995. ``What Numbers Could Be''. University of Pittsburgh, March.


%\end{revnumerate}

\bigskip

\marginhead{{\vskip 0.4em}Conferences Organized}
\medskip

\ind 2005. \emph{Frege, Identity, and Logic}, Institute of Philosophy in conjunction with the Mind Association. Papers by Ben Caplan, Richard Heck, Mark Eli Kalderon, Ian Rumfitt, Mark Sainsbury, Tom Smith, and Mark Textor.

\ind 2006. \emph{In Pursuit of Reason: Engaging Joseph Raz on Reason and Value}, Institute of Philosophy, University of London.

\ind 2012. \emph{Equality}, Institute of Philosophy, University of London. Papers by Niko Kolodny, Veronique Munoze-Darde, Joseph Raz, Samuel Scheffler.

\bigskip

\marginhead{{\vskip 0.4em}Grants}
\medskip
  
\ind 2004. AHRB Research Leave Scheme.

\bigskip 

\marginhead{{\vskip 0.9em}Administration}
\medskip

\ind 2011--present. Affiliate Tutor.

\ind 2004--11. Graduate Tutor

\ind 2007--11. Graduate Committee.

\ind 2004--8. Subject Panel in Philosophy

\ind 2002--4. Undergraduate Admissions Officer.

\ind 2001--present. Placement Officer.

\bigskip

\marginhead{{\vskip 0.9em}Knowledge \newline Transfer}
\medskip

\ind 2006--12. \emph{Excursus}, a blog on the technology of writing. Advocating the use of plain text over propriety binaries (such as Word documents) it presents a series of tutorials on how to adapt various tools for programmers to the task of academic writing.

\ind 2005--11. Editor of the Aristotelian Society. The objects of the Aristotelian Society are the advancement of public education in the field of philosophy and the publication of its proceedings to this end. 

\ind 2008. Wrote introductory remarks to \emph{Pleasure Island}, a collection of photographs by Jocelyn Bain Hogg of the Ibiza Rocks Festival.

\ind 2008. Interview with IC Radio on seeing red.

\ind 2007. Wrote and cowrote three articles for \emph{PracTeX} on using version control in the production of \LaTeX documents.

\ind 2006. Consultant for Ken Akita Design Office on philosophy and sustainable design.

\ind 2004-5. Taught seminars that allowed members of the public to discuss the papers presented to the Royal Institute of Philosophy.

\ind 2004. Consultant for Ken Akita Design Office on philosophy and sustainable design.

\ind 2001. Interview with the SciFi channel about Nietzsche and the nature of violence.

\bigskip

\marginhead{{\vskip 0.9em}Service to the \newline Profession}
\medskip


\ind Member (2010--present) Scientific Adviory Board for the The Center for the Senses.

\ind Editor (2005--11), \emph{Proceedings of the Aristotelian Society}.

\ind Editorial Board Member (2000--4), \emph{Mind}. 

\ind Reviewer, \emph{Mind}, \emph{Philosophical Studies},
\emph{The Philosophical Review}, \emph{Ratio}, \emph{Analysis}, \emph{The Review of Metaphysics}, \emph{Nous}, \emph{Pacific Philosophical Quarterly}.

\end{document}
